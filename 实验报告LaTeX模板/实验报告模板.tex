\documentclass[a4paper,12pt]{ctexart}
\usepackage[hmargin=1.25in,vmargin=1in]{geometry}
\usepackage{graphicx}
\usepackage{fancyhdr}
\usepackage{lastpage}
\usepackage{geometry}
\usepackage{setspace}
\usepackage{textcomp}
\usepackage{amsmath}
\usepackage{float}  %设置图片浮动位置的宏包
\usepackage{subfigure}  %插入多图时用子图显示的宏包
\usepackage{caption}
\usepackage{chemformula}
\usepackage{booktabs}
\usepackage{enumerate}
\usepackage{enumitem}
\usepackage{indentfirst}
\usepackage{cite}
\usepackage{url}

%页眉页脚设置
\usepackage{fancyhdr}
\pagestyle{fancy} 
\fancyhf{}
\fancyfoot[C]{\kaishu\thepage} 
\fancyhead[L]{\kaishu{中山大学物理与天文学院}} 
\fancyhead[R]{\kaishu{磁滞回线的测量}} 
\fancyfoot[R]{\kaishu{北落师门}}
\fancyfoot[L]{\kaishu{基础物理实验}}
\renewcommand{\headrulewidth}{0.4pt} 
\renewcommand{\footrulewidth}{0pt}


\begin{document}

%评分表格
\begin{table}[htbp]
    \centering
    \renewcommand\arraystretch{2}
    \setlength{\tabcolsep}{9mm}
    \begin{tabular}{|c|c|c|c|c|c|c|c|}

        \hline
        \multicolumn{2}{|c|}{预习报告}&
        \multicolumn{2}{c|}{实验记录}&
        \multicolumn{2}{c|}{分析讨论}&
        \multicolumn{2}{c|}{总成绩}\\
        \hline
            & & & & & & &  \\
        \hline
    \end{tabular}
\end{table}
%相关信息表格
\begin{table}[htbp]
	\centering
	\renewcommand\arraystretch{2}
	\setlength{\tabcolsep}{9mm}
	\begin{tabular}{|c|c|c|c|}
	\hline
	专业& 物理学类 & 年级 & 2020级 \\
	\hline
	姓名& 北落师门 & 学号 & 20201001 \\
	\hline
	室温&  & 实验地点 & 513 \\
	\hline
	日期& 2020年10月10日 & 教师签名& \\
	\hline
	\end{tabular}
\end{table}

\section*{实验BA5 磁滞回线测量}

\subsection*{【实验报告注意事项】}
\begin{enumerate}[itemsep=0pt,topsep=0pt]
	\item 实验报告由三部分组成
	\begin{enumerate}[itemsep=0pt,topsep=0pt]
		\item 预习报告:(提前一周)认真研读\textbf{实验讲义},弄清实验原理;
		实验所需的仪器设备、用具及其使用,完成课前预习思考题;了解实验
		需要测量的物理量,并根据要求提前准备实验记录表格。预习成绩低于
		10分(共20分)者不能做实验。
		\item 实验记录:认真、客观记录实验条件、实验过程中的现象以及
		数据。实验记录请用珠笔或者钢笔书写并签名(用铅笔记录的被认为无效)。
		保持原始记录,包括写错删除部分,如因误记需要修改记录,必须按规范修改。
		(不得输入电脑打印,但可扫描手记后打印扫描件);离开前请实验教师检查记录并签名。
		\item 分析讨论:处理实验原始数据(学习仪器使用类型的实验除外),对数据的可靠
		性和合理性进行分析;按规范呈现数据和结果(图、表),包括数据、图表按顺序编号
		及其引用;分析物理现象(含回答实验思考题,写出问题思考过程,必要时按规范引用数据);
		最后得出结论。
	\end{enumerate}
	\textbf{实验报告}就是将预习报告、实验记录、和数据处理与分析合起来,加上本页封面。
	\item 实验结束后完成实验报告,
	下次上课时交给带实验的教师(一周),最后一次实验结束一周后交\textbf{实验报告}。
	\item 除实验记录外,实验报告其他部分建议双面打印。
\end{enumerate}
\newpage

\section*{预习报告}


\newpage
\section*{实验记录}


\newpage
\section*{分析与讨论}








\end{document}

